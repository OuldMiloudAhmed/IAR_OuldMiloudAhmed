% !TEX TS-program = pdflatex
% !TEX encoding = UTF-8 Unicode

% This is a simple template for a LaTeX document using the "article" class.
% See "book", "report", "letter" for other types of document.

\documentclass[12pt]{article} % use larger type; default would be 10pt

\usepackage[utf8]{inputenc} % set input encoding (not needed with XeLaTeX)

%%% Examples of Article customizations
% These packages are optional, depending whether you want the features they provide.
% See the LaTeX Companion or other references for full information.

%%% PAGE DIMENSIONS
\usepackage[left=2.5cm,right=2cm,top=2cm,bottom=2.5cm]{geometry} % to change the page dimensions
\geometry{a4paper} % or letterpaper (US) or a5paper or....
% \geometry{margin=2in} % for example, change the margins to 2 inches all round
% \geometry{landscape} % set up the page for landscape
%   read geometry.pdf for detailed page layout information

\usepackage{graphicx} % support the \includegraphics command and options

% \usepackage[parfill]{parskip} % Activate to begin paragraphs with an empty line rather than an indent

%%% PACKAGES
\usepackage{booktabs} % for much better looking tables
\usepackage{array} % for better arrays (eg matrices) in maths
\usepackage{paralist} % very flexible & customisable lists (eg. enumerate/itemize, etc.)
\usepackage{verbatim} % adds environment for commenting out blocks of text & for better verbatim
\usepackage{subfig} % make it possible to include more than one captioned figure/table in a single float
% These packages are all incorporated in the memoir class to one degree or another...

%%% HEADERS & FOOTERS
\usepackage{fancyhdr} % This should be set AFTER setting up the page geometry
\pagestyle{empty} % options: empty , plain , fancy
\renewcommand{\headrulewidth}{0pt} % customise the layout...
\lhead{}\chead{}\rhead{}
\lfoot{}\cfoot{\thepage}\rfoot{}

%%% SECTION TITLE APPEARANCE
\usepackage{sectsty}
\allsectionsfont{\sffamily\mdseries\upshape} % (See the fntguide.pdf for font help)
% (This matches ConTeXt defaults)

%%% ToC (table of contents) APPEARANCE
\usepackage[nottoc,notlof,notlot]{tocbibind} % Put the bibliography in the ToC
\usepackage[titles,subfigure]{tocloft} % Alter the style of the Table of Contents
\renewcommand{\cftsecfont}{\rmfamily\mdseries\upshape}
\renewcommand{\cftsecpagefont}{\rmfamily\mdseries\upshape} % No bold!

%%% END Article customizations

%%% The "real" document content comes below...

\title{\textbf{Multi-Agent Modeling the Spread of COVID-19 epidemy}}
\author{Student : Ould Miloud Ahmed\\Email: ahmed.ouldmiloud@gmail.com\\Phone : 0665313319}
%\date{} % Activate to display a given date or no date (if empty),
         % otherwise the current date is printed 

\begin{document}
\maketitle
\section{Introduction: }
\hspace{5mm} During December 2019, a mystorious and contagious illness has been detected in \emph{Wuhan},\emph{ China }, due to The \emph{Sars-Cov-2} Coronavirus, Called later \emph {Covid-19}. Tow Month later, the rapid spread of coronavirus disease become a global threat affecting all most of countries in the world, what requires finding strategies and tools to monitor and prevent the spread of the disease. Multi-Agent modeling spread of an epidemy is one of this tool. 
\section{Why modeling an epidemic spread is important? }
\hspace{5mm}Using modeling help us to be able to do the following:\\1.	Inform our medical system about the number of patients expected in order to be prepared\\2.	Evaluate and modify public health measures \\3.	Predict a trajectory of an epidemic\\
4.	Try to understand how the disease spread’s\\

Although the models can’t predict what will happen, but they can expect the results of some scenarios, so we can plan and act to achieve the best possible results 


\section{Aim of the project: }
\hspace{5mm}In this work we use the concept of Multi-agent and simulation in order to conceive a multi-agent model who described the spread of covid-19. This model will be implemented in the Multi-agent platform JADE.
\section{Objectives: }
1.	Read some related researches of existing model in this area to better understand the topic\\
2.	Develop a model using a multi-agent platform JADE\\
3.	Test and evaluate some scenarios\\
4.	Complete the final report


\end{document}
